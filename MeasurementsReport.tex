\documentclass[conference, 11pt]{IEEEtran}
\IEEEoverridecommandlockouts
% The preceding line is only needed to identify funding in the first footnote. If that is unneeded, please comment it out.
\usepackage{cite}
\usepackage{amsmath,amssymb,amsfonts}
\usepackage{algorithmic}
\usepackage{graphicx}
\usepackage{textcomp}
\usepackage{xcolor}
\def\BibTeX{{\rm B\kern-.05em{\sc i\kern-.025em b}\kern-.08em
    T\kern-.1667em\lower.7ex\hbox{E}\kern-.125emX}}

%Removing indentation.
\newlength\tindent
\setlength{\tindent}{\parindent}
\setlength{\parindent}{0pt}
\renewcommand{\indent}{\hspace*{\tindent}}


\begin{document}

\title{Rain Detection Measurement System Using Infrared Transceiver\\
{\footnotesize \textsuperscript{}ELEN4006}
}

\author{\IEEEauthorblockN{1\textsuperscript{st} Darren Blanckensee}
\IEEEauthorblockA{\textit{School of Electrical and Information Engineering} \\
\textit{University of the Witwatersrand}\\
Johannesburg, South Africa \\
1147279@students.wits.ac.za}
\and
\IEEEauthorblockN{2\textsuperscript{nd} Uyanda Mphunga}
\IEEEauthorblockA{\textit{School of Electrical and Information Engineering} \\
\textit{University of the Witwatersrand}\\
Johannesburg, South Africa \\
1168101@students.wits.ac.za}
}

\maketitle

\begin{abstract}

\end{abstract}

\begin{IEEEkeywords}

\end{IEEEkeywords}

\section{Introduction}
\IEEEPARstart{T}{he} purpose of this project is design and simulate a windshield moisture detection measurement system for an electric vehicle. The system is meant to produce produce an output with information on the amount of moisture on the windshield. This information can then be sent to other systems of the vehicle. The measurement system's design is based on fundamental principles of measurement systems. This report includes sections on the general characteristics of a measurement system, Static \& Dynamic Characteristics, Noise, Sensing Element, Conditioning Element, Simulation Results and a Strengths Weaknesses Advantages and Threats~(SWAT) evaluation of the measurement system.

\section{Background}
	
	\subsection{Measurement Systems}
	
	\subsection{Literature Survey}
	
	\subsection{Optics (Refraction Business)}
	
	\subsection{Real World Simulation (Noise business and real components)}
	

\section{Design}

\section{Simulation Results}

\section{SWAT Analysis}


\section{Conclusion}

\end{document}
